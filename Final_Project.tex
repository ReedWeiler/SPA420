% Options for packages loaded elsewhere
\PassOptionsToPackage{unicode}{hyperref}
\PassOptionsToPackage{hyphens}{url}
%
\documentclass[
]{article}
\title{Personal Income and COVID-19 Vaccination Rates}
\author{-Na Li\footnote{American University} -Reed Weiler\footnote{American
  University} -Yun-Jen Tsai\footnote{American University}}
\date{2022-01-09}

\usepackage{amsmath,amssymb}
\usepackage{lmodern}
\usepackage{iftex}
\ifPDFTeX
  \usepackage[T1]{fontenc}
  \usepackage[utf8]{inputenc}
  \usepackage{textcomp} % provide euro and other symbols
\else % if luatex or xetex
  \usepackage{unicode-math}
  \defaultfontfeatures{Scale=MatchLowercase}
  \defaultfontfeatures[\rmfamily]{Ligatures=TeX,Scale=1}
\fi
% Use upquote if available, for straight quotes in verbatim environments
\IfFileExists{upquote.sty}{\usepackage{upquote}}{}
\IfFileExists{microtype.sty}{% use microtype if available
  \usepackage[]{microtype}
  \UseMicrotypeSet[protrusion]{basicmath} % disable protrusion for tt fonts
}{}
\makeatletter
\@ifundefined{KOMAClassName}{% if non-KOMA class
  \IfFileExists{parskip.sty}{%
    \usepackage{parskip}
  }{% else
    \setlength{\parindent}{0pt}
    \setlength{\parskip}{6pt plus 2pt minus 1pt}}
}{% if KOMA class
  \KOMAoptions{parskip=half}}
\makeatother
\usepackage{xcolor}
\IfFileExists{xurl.sty}{\usepackage{xurl}}{} % add URL line breaks if available
\IfFileExists{bookmark.sty}{\usepackage{bookmark}}{\usepackage{hyperref}}
\hypersetup{
  pdftitle={Personal Income and COVID-19 Vaccination Rates},
  pdfauthor={-Na Li -Reed Weiler -Yun-Jen Tsai},
  hidelinks,
  pdfcreator={LaTeX via pandoc}}
\urlstyle{same} % disable monospaced font for URLs
\usepackage[margin=1in]{geometry}
\usepackage{color}
\usepackage{fancyvrb}
\newcommand{\VerbBar}{|}
\newcommand{\VERB}{\Verb[commandchars=\\\{\}]}
\DefineVerbatimEnvironment{Highlighting}{Verbatim}{commandchars=\\\{\}}
% Add ',fontsize=\small' for more characters per line
\usepackage{framed}
\definecolor{shadecolor}{RGB}{248,248,248}
\newenvironment{Shaded}{\begin{snugshade}}{\end{snugshade}}
\newcommand{\AlertTok}[1]{\textcolor[rgb]{0.94,0.16,0.16}{#1}}
\newcommand{\AnnotationTok}[1]{\textcolor[rgb]{0.56,0.35,0.01}{\textbf{\textit{#1}}}}
\newcommand{\AttributeTok}[1]{\textcolor[rgb]{0.77,0.63,0.00}{#1}}
\newcommand{\BaseNTok}[1]{\textcolor[rgb]{0.00,0.00,0.81}{#1}}
\newcommand{\BuiltInTok}[1]{#1}
\newcommand{\CharTok}[1]{\textcolor[rgb]{0.31,0.60,0.02}{#1}}
\newcommand{\CommentTok}[1]{\textcolor[rgb]{0.56,0.35,0.01}{\textit{#1}}}
\newcommand{\CommentVarTok}[1]{\textcolor[rgb]{0.56,0.35,0.01}{\textbf{\textit{#1}}}}
\newcommand{\ConstantTok}[1]{\textcolor[rgb]{0.00,0.00,0.00}{#1}}
\newcommand{\ControlFlowTok}[1]{\textcolor[rgb]{0.13,0.29,0.53}{\textbf{#1}}}
\newcommand{\DataTypeTok}[1]{\textcolor[rgb]{0.13,0.29,0.53}{#1}}
\newcommand{\DecValTok}[1]{\textcolor[rgb]{0.00,0.00,0.81}{#1}}
\newcommand{\DocumentationTok}[1]{\textcolor[rgb]{0.56,0.35,0.01}{\textbf{\textit{#1}}}}
\newcommand{\ErrorTok}[1]{\textcolor[rgb]{0.64,0.00,0.00}{\textbf{#1}}}
\newcommand{\ExtensionTok}[1]{#1}
\newcommand{\FloatTok}[1]{\textcolor[rgb]{0.00,0.00,0.81}{#1}}
\newcommand{\FunctionTok}[1]{\textcolor[rgb]{0.00,0.00,0.00}{#1}}
\newcommand{\ImportTok}[1]{#1}
\newcommand{\InformationTok}[1]{\textcolor[rgb]{0.56,0.35,0.01}{\textbf{\textit{#1}}}}
\newcommand{\KeywordTok}[1]{\textcolor[rgb]{0.13,0.29,0.53}{\textbf{#1}}}
\newcommand{\NormalTok}[1]{#1}
\newcommand{\OperatorTok}[1]{\textcolor[rgb]{0.81,0.36,0.00}{\textbf{#1}}}
\newcommand{\OtherTok}[1]{\textcolor[rgb]{0.56,0.35,0.01}{#1}}
\newcommand{\PreprocessorTok}[1]{\textcolor[rgb]{0.56,0.35,0.01}{\textit{#1}}}
\newcommand{\RegionMarkerTok}[1]{#1}
\newcommand{\SpecialCharTok}[1]{\textcolor[rgb]{0.00,0.00,0.00}{#1}}
\newcommand{\SpecialStringTok}[1]{\textcolor[rgb]{0.31,0.60,0.02}{#1}}
\newcommand{\StringTok}[1]{\textcolor[rgb]{0.31,0.60,0.02}{#1}}
\newcommand{\VariableTok}[1]{\textcolor[rgb]{0.00,0.00,0.00}{#1}}
\newcommand{\VerbatimStringTok}[1]{\textcolor[rgb]{0.31,0.60,0.02}{#1}}
\newcommand{\WarningTok}[1]{\textcolor[rgb]{0.56,0.35,0.01}{\textbf{\textit{#1}}}}
\usepackage{graphicx}
\makeatletter
\def\maxwidth{\ifdim\Gin@nat@width>\linewidth\linewidth\else\Gin@nat@width\fi}
\def\maxheight{\ifdim\Gin@nat@height>\textheight\textheight\else\Gin@nat@height\fi}
\makeatother
% Scale images if necessary, so that they will not overflow the page
% margins by default, and it is still possible to overwrite the defaults
% using explicit options in \includegraphics[width, height, ...]{}
\setkeys{Gin}{width=\maxwidth,height=\maxheight,keepaspectratio}
% Set default figure placement to htbp
\makeatletter
\def\fps@figure{htbp}
\makeatother
\setlength{\emergencystretch}{3em} % prevent overfull lines
\providecommand{\tightlist}{%
  \setlength{\itemsep}{0pt}\setlength{\parskip}{0pt}}
\setcounter{secnumdepth}{5}
\ifLuaTeX
  \usepackage{selnolig}  % disable illegal ligatures
\fi

\begin{document}
\maketitle

\hypertarget{introduction}{%
\section{Introduction}\label{introduction}}

There is a growing media consensus that income might have something to
do with vaccination rates, so we wanted to investigate whether there is
any merit to these claims. Research on this topic could potentially aid
in understanding how economic inequality poses a barrier to effective
strategies for combating the pandemic, and inform future pandemic
response strategies.

\hypertarget{our-substance-and-context-section-title-here}{%
\section{{[}Our Substance and Context Section Title
Here{]}}\label{our-substance-and-context-section-title-here}}

(Jenell) Talk about why we choose the two data sets and what methods
that we plan to use to see the relations that we are trying to describe.

(Emphasize personal health benefits to boost COVID-19 vaccination rates:
\url{https://www.pnas.org/content/pnas/118/32/e2108225118.full.pdf})

(Correlation Between Health and Wealth:
\url{https://militaryfamilieslearningnetwork.org/2019/08/08/correlations-between-health-and-wealth/})

\hypertarget{data-and-methods}{%
\section{Data and Methods}\label{data-and-methods}}

\label{section:data}

Data is drawn from two separate sources. One data set contains personal
income level (GDP) by state, the other contains total vaccination rates
by state. Our combined dataset has 51 observations, representing each of
the 50 united states, and 1 district.

We chose to use a simple linear regression to test the relationship
between income per capita and COVID-19 vaccination rates at a state
level within the United States. In this analysis, the explanatory
variable was income level, and the outcome variable was COVID-19
vaccination rates. We created a scatter plot to model the relationship
between income and vaccination rate by state, displayed below.

\begin{Shaded}
\begin{Highlighting}[]
\FunctionTok{library}\NormalTok{(readxl)}
\NormalTok{income }\OtherTok{\textless{}{-}} \FunctionTok{read\_excel}\NormalTok{(}\StringTok{"income Ratio.xlsx"}\NormalTok{)}
\FunctionTok{colnames}\NormalTok{(income) }\OtherTok{\textless{}{-}} \FunctionTok{c}\NormalTok{(}\StringTok{"State.Name"}\NormalTok{, }\StringTok{"Income"}\NormalTok{)}
\FunctionTok{head}\NormalTok{(income)}
\end{Highlighting}
\end{Shaded}

\begin{verbatim}
## # A tibble: 6 x 2
##   State.Name Income
##   <chr>       <dbl>
## 1 Alabama    0.0493
## 2 Alaska     0.0758
## 3 Arizona    0.0563
## 4 Arkansas   0.0483
## 5 California 0.0855
## 6 Colorado   0.0732
\end{verbatim}

\begin{Shaded}
\begin{Highlighting}[]
\NormalTok{vaccination }\OtherTok{\textless{}{-}} \FunctionTok{read.csv}\NormalTok{(}\StringTok{"vaccination ratio.csv"}\NormalTok{, }\AttributeTok{fileEncoding =} \StringTok{"UTF{-}8{-}BOM"}\NormalTok{)}
\FunctionTok{head}\NormalTok{(vaccination)}
\end{Highlighting}
\end{Shaded}

\begin{verbatim}
##   State.Name Vaccination.Ratio
## 1    Alabama              1.66
## 2     Alaska              1.83
## 3    Arizona              1.78
## 4   Arkansas              1.72
## 5 California              1.98
## 6   Colorado              1.91
\end{verbatim}

\begin{Shaded}
\begin{Highlighting}[]
\NormalTok{dat }\OtherTok{\textless{}{-}} \FunctionTok{merge}\NormalTok{(income, vaccination)}
\FunctionTok{head}\NormalTok{(dat)}
\end{Highlighting}
\end{Shaded}

\begin{verbatim}
##   State.Name     Income Vaccination.Ratio
## 1    Alabama 0.04931529              1.66
## 2     Alaska 0.07579234              1.83
## 3    Arizona 0.05628906              1.78
## 4   Arkansas 0.04834675              1.72
## 5 California 0.08546529              1.98
## 6   Colorado 0.07322607              1.91
\end{verbatim}

\hypertarget{our-results-section-title-here}{%
\section{{[}Our Results Section Title
Here{]}}\label{our-results-section-title-here}}

Here, we explain and interpret our results. We try to learn as much as
we can about our question as possible, given the data and analysis. We
present our results clearly. We interpret them for the reader with
precision and circumspection. We avoid making claims that are not
substantiated by our data.

Note that this section may be integrated into Section
\ref{section:data}, if joining the two improves the overall
presentation.

Our results for the \texttt{cars} data include estimating the linear
model

\[\text{Vaccination.Ratio}_i = \beta_0 + \beta_1 (\text{Speed}_i) + \epsilon_i.\]

\begin{Shaded}
\begin{Highlighting}[]
\FunctionTok{plot}\NormalTok{(Vaccination.Ratio}\SpecialCharTok{\textasciitilde{}}\NormalTok{Income, dat)}
\FunctionTok{abline}\NormalTok{(lm\_out)}
\end{Highlighting}
\end{Shaded}

\includegraphics{Final_Project_files/figure-latex/unnamed-chunk-5-1.pdf}

\begin{table}[!htbp] \centering 
  \caption{regression result} 
  \label{} 
\begin{tabular}{@{\extracolsep{5pt}}lc} 
\\[-1.8ex]\hline 
\hline \\[-1.8ex] 
 & \multicolumn{1}{c}{Vaccination ratio} \\ 
\cline{2-2} 
\\[-1.8ex] & Vaccination.Ratio \\ 
\hline \\[-1.8ex] 
 Income & 4.06$^{***}$ \\ 
  & (1.03) \\ 
  & \\ 
 Constant & 1.59$^{***}$ \\ 
  & (0.07) \\ 
  & \\ 
\hline \\[-1.8ex] 
Observations & 51 \\ 
R$^{2}$ & 0.24 \\ 
Adjusted R$^{2}$ & 0.23 \\ 
Residual Std. Error & 0.19 (df = 49) \\ 
F Statistic & 15.60$^{***}$ (df = 1; 49) \\ 
\hline 
\hline \\[-1.8ex] 
\textit{Note:}  & \multicolumn{1}{r}{$^{*}$p$<$0.1; $^{**}$p$<$0.05; $^{***}$p$<$0.01} \\ 
\end{tabular} 
\end{table}

\hypertarget{discussion}{%
\section{Discussion}\label{discussion}}

One limitation of our data is that the CDC doesn't specify whether the
``total vaccination'' data counts each dose of the vaccine as a separate
instance of ``vaccination,'' nor does it specify how it would account
for booster shots, etc. So, there is some ambiguity as to how to
interpret the results of our analysis, since the outcome variable could
potentially be measuring a variety of different scenarios. Future
research should strive to distinguish between instances of single-dose
vaccination, double-dose vaccination, inclusion of a booster shot,
and/or non-vaccination status as separate categories so as to better
isolate the statistical impact of income level on each distinct outcome.

\hypertarget{references}{%
\section{References}\label{references}}

\end{document}
